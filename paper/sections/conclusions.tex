\section{Conclusions}\label{section:conclusion}

% \begin{itemize}
%     \item Observations
%     \item Threads to validity
%     \item Future Work
% \end{itemize}

In this paper, we presented a tool that automates the traceability link recovery by identifying trace links between requirements written in natural language and software development artifacts(Issues, PRs, Commits) obtained from GitHub repositories. Through the implementation of three optional methods, namely keyword extraction, TF-IDF vector and word vector, we have demonstrated the capability of identifying traceability links of our tool. We have stored the traceability data we obtained in a graph database to visualize the trace links, and supported the data with an interactive dashboard, which provides users with comprehensive statistical insights.
Our experiments on the evaluation of the tool have led us to observe that the quality and consistency of the requirement statements and the software development artifacts(SDA) significantly impact the effectiveness of the tool's capability of identifying trace links. Well-written requirement specifications that are self-explanatory and granulated lead to easier identification of its traces along with SDA which have textual data (e.g. title, description, comments) that is semantically related to the feature they implement. In this manner, our tool could be utilized for educational purposes, to assist in the improvement of writing requirement specifications.

It is important to acknowledge the threats to the validity of our study. Firstly, the absence of a ground truth set that is established by analysts may introduce bias in our evaluation process. We have prepared the ground truth set and conducted our experiments on the student projects we have participated in. Thus, we do know the capabilities of our tool and the project that we experimented on. Future work should involve validation with industry professionals and experiments designed with a wider range of projects to enhance the operating scale of our tool. 

Other several avenues for future work include providing trace links between different kinds of software artifacts, such as from Issues to PRs, to enable a deeper view of the development process. Additionally, we plan to integrate different types of requirement specifications, to accommodate diverse requirement specification document formats. Furthermore, we intend to develop a product that fetches software artifacts from various platforms such as GitLab and Jira, to improve the compatibility of the tool with different development environments that are mostly preferred within the industry. Finally, we would like to expand the variety of methods used in identifying trace links based on the needs of the users and improve the accuracy.
% by increasing the recall and precision values obtained.

In conclusion, our tool provides an automated solution for requirement traceability(RT) and enhances the management of software projects. By optimizing the time and effort required for RT and offering educational opportunities for project management, our tool contributes to the overall success of software development projects. With future development and evaluation, our study carries the potential to positively impact traceability practices.



%We present a shared collaborative modeling environment instrumented with an automated glossary extraction pipeline from model labels using state-of-the-art NLP techniques without any supervision. Our approach prevents stakeholders from spending effort on a challenging and time-consuming task of manual glossary extraction. We develop two heuristics that use pre-trained BERT models and one simple heuristic that depends on substring matching.

 %In this section, we share our plans for glossary extraction. First, we plan to conduct an experimental evaluation with human experts as we stated in Sect. \ref{section:evaluation}. Second, we plan to improve our glossary extraction heuristics' performance using human experts' feedbacks. Finally, we aim to extend our method for automated domain model extraction from model labels and employing our method as a complementary for existing systems to form a hybrid system.
