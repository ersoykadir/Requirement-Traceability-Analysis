\section{Conclusions}
\label{sec:conc}

This paper aims to visualize software repositories through the trace links from requirements to other software development artifacts that are issues, pull requests, and commits. We extract trace links using three different methods: \emph{i.} keyword matching, \emph{ii.} TF-IDF vectors, and \emph{iii.} word vectors and store the repository information and traces in a graph database. We query the database to visualize the information from the repository such as number of traces of a requirement per week, traces of a requirement, created and closed issues, and other statistics in a dashboard to have a better understanding of the development process. We present the results of a preliminary evaluation on the performance of trace link extraction methods and our future evaluation plans. We envision our dashboard is used for educational purposes as well as in the industry.

Our immediate future work includes a thorough evaluation of the performance of the trace link extraction methods and the usefulness and usability of our dashboard. The approach itself can be extending by incorporating other artifacts to be analysed from the repositories. The performance of the techniques can be enhances by using bigger and software related corpora for word-vector method. Our own keyword matching technique can be compared against large language models for the same task.

%In this paper, we presented a tool that automates the traceability link recovery by identifying trace links between requirements written in natural language and software development artifacts(Issues, PRs, Commits) obtained from GitHub repositories. Through the implementation of three optional methods, namely keyword extraction, TF-IDF vector and word vector, we have demonstrated the capability of identifying traceability links of our tool. We have stored the traceability data we obtained in a graph database to visualize the trace links, and supported the data with an interactive dashboard, which provides users with comprehensive statistical insights.

%In conclusion, our tool provides an automated solution for requirement traceability(RT) and enhances the management of software projects. By optimizing the time and effort required for RT and offering educational opportunities for project management, our tool contributes to the overall success of software development projects. With future development and evaluation, our study carries the potential to positively impact traceability practices.

%Other several avenues for future work include providing trace links between different kinds of software artifacts, such as from Issues to PRs, to enable a deeper view of the development process. Additionally, we plan to integrate different types of requirement specifications, to accommodate diverse requirement specification document formats. Furthermore, we intend to develop a product that fetches software artifacts from various platforms such as GitLab and Jira, to improve the compatibility of the tool with different development environments that are mostly preferred within the industry. Finally, we would like to expand the variety of methods used in identifying trace links based on the needs of the users and improve the accuracy.

%%% Local Variables:
%%% mode: latex
%%% TeX-master: "../main"
%%% End: