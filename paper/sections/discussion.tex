\section{Discussion}\label{section:discussion}

Our experiments on the evaluation of the tool have led us to observe that the quality and consistency of the requirement statements and the software development artifacts(SDA) significantly impact the effectiveness of the tool's capability of identifying trace links. Well-written requirement specifications that are self-explanatory and granulated lead to easier identification of its traces along with SDA which have textual data (e.g. title, description, comments) that is semantically related to the feature they implement. In this manner, our tool could be utilized for educational purposes, to assist in the improvement of writing requirement specifications.

It is important to acknowledge the threats to the validity of our study. Firstly, the absence of a ground truth set that is established by analysts may introduce bias in our evaluation process. We have prepared the ground truth set and conducted our experiments on the student projects we have participated in. Thus, we do know the capabilities of our tool and the project that we experimented on. Future work should involve validation with industry professionals and experiments designed with a wider range of projects to enhance the operating scale of our tool. 

Other several avenues for future work include providing trace links between different kinds of software artifacts, such as from Issues to PRs, to enable a deeper view of the development process. Additionally, we plan to integrate different types of requirement specifications, to accommodate diverse requirement specification document formats. Furthermore, we intend to develop a product that fetches software artifacts from various platforms such as GitLab and Jira, to improve the compatibility of the tool with different development environments that are mostly preferred within the industry. Finally, we would like to expand the variety of methods used in identifying trace links based on the needs of the users and improve the accuracy.
