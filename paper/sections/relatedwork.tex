\section{Related Work}
\label{section:related_work}

There are several studies that focus on automatizing requirements traceability by implementing various techniques. Hayes et al. (2003)\cite{hayes-2003} have studied an approach that frames requirement traceability as an information retrieval(IR) problem. Their study aims to enhance recall and precision values of capturing traces. The IR methods they have implemented performed a significantly higher retrieval percentage, relative to the manual tracing. In a different study, Abdeen (2023)\cite{abdeen-2023} established a system that grips taxonomic trace links to link software development artifacts and requirements. In this approach, requirement artifacts are labeled with system-recommended labels based on their textual data. Bonner et al. (2023)\cite{bonner-2023} have developed a tool that performs Artificial Intelligence(AI) to identify trace links between requirements and model-based designs. They have integrated their tool into the Siemens toolchain for Application Lifecycle Management(ALM) and performed experiments on the traceability problem with the natural language requirements and system design models. Cleland-Huang et al. (2007)\cite{cleland-huang-2007} focused on nine best practices for implementing automated traceability. These practices include information retrieval methods, and they have significantly diminished the amount of effort given to produce a requirement trace matrix. Mills (2017)\cite{mills-2017} automated traceability link recovery(TLR) with machine learning algorithms and approached TLR as a binary classification problem.

These studies have integrated relatively more modern technologies such as artificial intelligence, machine learning, and information retrieval algorithms to automated trace links recovery and encouraged the researches to implement various approaches to minimize the effort and maximize the quality of requirement traceability.

