\documentclass[conference]{IEEEtran}
\usepackage[utf8]{inputenc} % To use Unicode (e.g. Turkish) characters
\usepackage[T1]{fontenc}
\renewcommand{\labelenumi}{(\roman{enumi})}
\usepackage{amsmath, amsthm, amssymb}
 % Some extra symbols
\usepackage[bottom]{footmisc}
\usepackage{cite}
\usepackage{graphicx}
\usepackage{url}
\usepackage{multicol}
\usepackage{longtable}
\usepackage{xargs}
\usepackage{booktabs}
%
\usepackage[colorinlistoftodos,prependcaption,textsize=tiny]{todonotes}
\newcommandx{\unsure}[2][1=]{\todo[linecolor=red,backgroundcolor=red!25,bordercolor=red,#1]{#2}}
\newcommandx{\change}[2][1=]{\todo[linecolor=blue,backgroundcolor=blue!25,bordercolor=blue,#1]{#2}}
\newcommandx{\info}[2][1=]{\todo[linecolor=OliveGreen,backgroundcolor=OliveGreen!25,bordercolor=OliveGreen,#1]{#2}}
\newcommandx{\improvement}[2][1=]{\todo[linecolor=Plum,backgroundcolor=Plum!25,bordercolor=Plum,#1]{#2}}
\newcommandx{\thiswillnotshow}[2][1=]{\todo[disable,#1]{#2}}
%
%\usepackage{dingbat}
\graphicspath{{figs/}} % Graphics will be here

\usepackage{multirow}
\usepackage{subcaption}
\usepackage[linesnumbered,ruled,vlined]{algorithm2e}

\usepackage{soul}
\usepackage{adjustbox}
\usepackage{svg}
%\usepackage[dvipsnames]{xcolor}
\DeclareRobustCommand{\fba}[1]{ {\begingroup\sethlcolor{BurntOrange}\hl{(fba:) #1}\endgroup} }

\begin{document}


\title{Extracting Trace Links in Software Repositories}
% \author{
%  \IEEEauthorblockN{Kadir Ersoy, Ecenur Sezer, Suzan Üsküdarlı, Fatma Başak Aydemir}
% \IEEEauthorblockA{
% Boğaziçi University\\
% Istanbul, Türkiye \\
% \{kadir.ersoy, ecenur.sezer, suzan.uskudarli, basak.aydemir\}@boun.edu.tr}
% }

\author{
  \IEEEauthorblockN{Anonymous Authors}\\
\IEEEauthorblockA{Anonymous}}

%\date{May 2021}

\maketitle
\begin{abstract}
  % context
 Requirements traceability refers to the capability of following the life of a requirement in both forwards and backward directions, and to the ability to link requirements to other software artifacts through particular relationships.
  % Motivation
These traceability links enable stakeholders to monitor the development progress of requirements and assist system engineers in tracking the effort and workload associated with fulfilling those requirements.
  % Approach
Traditional methods to recover traceability links necessitate significant human effort, making them unsuitable and inefficient, particularly as the project scale grows.
  % Results
In this work, we present a tool that establishes automated requirement traceability links between requirements written in natural language and software artifacts acquired from GitHub repositories. The tool implements three optional methods to trace the artifacts to the requirements, namely a keyword extraction pipeline, TF-IDF vector and word-vector. The captured traces are stored in a graph database and visualized. Additionally, an interactive dashboard featuring statistical data about the artifacts and their traces is implemented. By offering automated traceability and comprehensive visualization, this tool aims to enhance the management of requirements and to understand their quality in software development projects.
  %Contribution

\end{abstract}
\begin{IEEEkeywords}
requirements traceability, natural language processing,  keyword extraction, trace graph, software management
\end{IEEEkeywords}


\section{Introduction}
\label{sec:intro}

% \begin{itemize}
%     \item Requirements traceability definition
%     \item Why do we do it, examples
%     \begin{itemize}
%         \item It is too hard to do it manually
%         \begin{itemize}
%             \item Cost of integration: Need trace link recovery,
%             \item Human effort: Do not force extra measures for traceability
%         \end{itemize}
%         \item Its role in project management
%         \begin{itemize}
%             \item Monitoring the progress: Status of the project
%             \item Effort assessment: How much effort went to which features
%             \item Detecting problems for specific features
%         \end{itemize}
%     \end{itemize}
%     \item showcasing trace data and other stats to user in a self-evident manner
%     \end{itemize}

%     Requirements trace

%Yazılım işleri karışık çok fazla by-product var / TRace linkler hayatı kolaylaştırır / ama trace'leri de manual yapmak time and effort/ automated trace link recovery bize performans ve zaman kazandırıyor/ bu sebeple biz de şöyle şöyle bir iş öneriyoruz / paper structure

%
%Software development is a complex process where numerous artifacts are produced as the software requirements are designed, implemented, and tested. These artifacts, such as commits, issues, and pull requests, are related to one or more requirements and indicate the progress of the project. We need to trace the requirements through these artifact, aggregate and visualize the results to understand the current status of the project. However, due to their sheer volume and dynamic nature, attempting to manually track and link these artifacts to specific software requirements and understanding the status of the project can be a daunting task. Automated approaches provide a more efficient software requirements traceability.

%
The success of a software development project is determined by whether the requirements are satisfied or not. Project managers need to check the progress throughout the development cycle to ensure the timely delivery of requirements. Software development is a complex process where numerous artifacts are produced. These artifacts, such as commits, issues, and pull requests, are related to one or more requirements and traceability of requirements~\cite{gotel-1994} to these artifacts indicate the progress of the project. However, due to their sheer volume and dynamic nature, attempting to manually track and link these artifacts to specific software requirements and understanding the status of the project can be a daunting task.

%
Emerging automated approaches are used provide a more efficient software requirements traceability \cite{cleland-huang-2007,mills-2017,VANOOSTEN2023107226,bonner-2023,deen-2023}. Natural language processing techniques, together with information retrieval and machine learning methods increase the time efficiency of the task of requirements tracing~\cite{cleland-huang-2007}.

%
This paper aims at visualizing software development repositories to support real-time and retropective analysis of software development pojects based on the trace links between requirements and other software development artifacts. We implement a dashboard aggregating several visual and statistical data of the project and specific requirement progress through the trace links. We analyze a given software repository and extract directed trace links from requirements to issues, pull requests, and commits using the textual attributes of these software development artifacts such as summary and message. We implement and evaluate three methods to extract trace links: \emph{i.} keyword matching, \emph{ii.} TF-IDF vectors, \emph{iii.} word vectors. We store the artifacts together with their attributes and the trace links among them in a graph database and use this database to visualize the repository for software management.

The main contribution of this paper is the dashboard designed to support software development management. The dashboard presents data visualizations of repository statistics, requirements trace links, and their temporal distribution to support real-time and retrospective analysis of software repositories. We publicly share our implementation and evaluation data in our replication package.















%Requirements traceability refers to the capability of following the life of a requirement in both forwards and backward directions~\cite{gotel-1994}, and to the ability to link requirements to other software artifacts through particular relationships. Requirement traceability is crucial in software development projects since it shows the progress done in implementing the requirements and assists engineers in observing the effort and workload associated with them. All these aspects of requirement traceability help teams to ensure the fulfillment of specified requirements, by linking them to other software artifacts. Given that, it is a critical obstacle that traditional methods of recovering traceability links require significant time and human effort, making them unsuitable for the task as the project scale grows. Moreover, the quality of these methods is not consistent, since it highly depends on the understanding of the requirement traceability of the human analysts' performing them.
%In this paper, we present a tool implemented to automate the process of identifying trace links. Our tool establishes links from requirements that are written in natural language to the software development artifacts, mainly Issues, PRs, and Commits, that are fetched from GitHub repositories. By automatizing this process, our tool aims to decrease the reliance on human effort and provides efficient requirement traceability management.

%For identifying the trace links, three optional methods are implemented in our tool. These methods are keyword extraction, which performs keyword matching to software development artifacts, and TF-IDF Vector - Word Vector methods, which basically perform vector-based similarity. Each of these methods offers distinct advantages that are highly dependent on the context of the projects.
%The software artifacts and the identified trace links are stored in a graph database and visualized in a graphical structure to enable efficient retrieval of traceability information. To improve the comprehension of the collected traceability data, our tool also features an interactive dashboard. This dashboard is equipped with valuable statistical insights about the lifetime of the project and the requirements. Along with the visualization of the traceability links and the offering of statistical data, we aim to improve the management of the requirements, ultimately leading to better-planned and programmed software projects.

The rest of the paper is structured as follows. Sec.~\ref{sec:relwork} presents the related work from the literature. Sec.~\ref{sec:approach} details our approach for identifying trace links and selected information visualization for our dashboard. Our preliminary evaluation, and future evaluation plans are described in Sec.~\ref{sec:eval}. Sec.~\ref{sec:discuss} discusses our observations, limitations of our work and threats to validity. Finally, Sec.~\ref{sec:conc} concludes the paper and lays out the future work.

%\section{Problem and Motivation}
%\label{section:problem}

% Requirements traceability,
%     - definition,
% Why do we do it, examples
%     - It is too hard to do it manually
%     - Its role in project management,
%         - Monitoring the progress
%             - Status of the project
%             - What has been done
%         - Effort assessment
%             - How much effort went to which features
%         - Detecting problems for specific features
%             - In the case of a problem, look for it in a localized manner
%             - can be done since we know what has done for which context
%     - The use of requirements and req traceability is not common
%         - Cost of integration
%             - Need trace link recovery,
%         - Human effort
%             - Do not force extra measures for traceability
%     - To achieve wide-use of traceability
%         - user friendly,
%             - showcasing trace data and other stats to user in a self-evident manner
%         - minimal human effort,
%%% Local Variables:
%%% mode: latex
%%% TeX-master: "../main"
%%% End:
\section{Related Work}
\label{section:related_work}

There are several studies that focus on automatizing requirements traceability by implementing various techniques. Hayes et al. (2003)\cite{hayes-2003} have studied an approach that frames requirement traceability as an information retrieval(IR) problem. Their study aims to enhance recall and precision values of capturing traces. The IR methods they have implemented performed a significantly higher retrieval percentage, relative to the manual tracing. In a different study, Abdeen (2023)\cite{abdeen-2023} established a system that grips taxonomic trace links to link software development artifacts and requirements. In this approach, requirement artifacts are labeled with system-recommended labels based on their textual data. Bonner et al. (2023)\cite{bonner-2023} have developed a tool that performs Artificial Intelligence(AI) to identify trace links between requirements and model-based designs. They have integrated their tool into the Siemens toolchain for Application Lifecycle Management(ALM) and performed experiments on the traceability problem with the natural language requirements and system design models. Cleland-Huang et al. (2007)\cite{cleland-huang-2007} focused on nine best practices for implementing automated traceability. These practices include information retrieval methods, and they have significantly diminished the amount of effort given to produce a requirement trace matrix. Mills (2017)\cite{mills-2017} automated traceability link recovery(TLR) with machine learning algorithms and approached TLR as a binary classification problem.

These studies have integrated relatively more modern technologies such as artificial intelligence, machine learning, and information retrieval algorithms to automated trace links recovery and encouraged the researches to implement various approaches to minimize the effort and maximize the quality of requirement traceability.


\section{Method}
\label{sec:approach}

This section details our approach to discovering trace links in a software repository. Our approach takes a software repository and requirements as input and extracts trace links between requirements and the software issues, commits, and pull requests (PRs) by analyzing the textual fields of these artifacts. Our prototype tool visualizes the trace links and other information on the repository. Figure~\ref{fig:sys-flow} presents the main steps of our approach. We publicly share the implementation of our approach in our replication package\footnote{https://zenodo.org/record/8076982}

\begin{figure}[htb]
    \centering
    \includegraphics[width=0.65\linewidth]{figs/approach.png}
    \caption{Steps of our approach}
    \label{fig:sys-flow}
  \end{figure}

  The first two steps, \textsf{S1} and \textsf{S2}, concern processing the two inputs of our approach, a software repository and natural language requirements, respectively. \textsf{S1} fetches issues, pull requests, and commits from a repository whose URL is given. Our prototype expects a GitHub repository, however, our approach is general and can be applied to other repositories where the aforementioned development artifacts are present. %Table~\ref{tab:artifactfeatures} presents the attributes of these artifacts that are used in our approach.
  Our prototype expects a text file containing requirements. It does not enforce specific requirements and is able to process the requirements that are written in a hierarchical structure, which is a common practice.

          \begin{table}
        \centering
        \caption{Attributes of software development artifacts used in our approach}
        \label{tab:artifactfeatures}
        \begin{tabular}{lllll}
          \toprule
          & Requirement & Issue & PR & Commit \\
          \midrule
          ID &\checkmark &\checkmark&\checkmark&\checkmark\\
          Title &-&\checkmark&\checkmark&-\\
          Description &\checkmark&\checkmark&\checkmark&-\\
          URL&-&\checkmark&\checkmark&\checkmark\\
          Number&\checkmark&\checkmark&\checkmark&\checkmark\\
          State&-&\checkmark&\checkmark&-\\
          Creation Date&-&\checkmark&\checkmark&-\\
          Completion Date&-&\checkmark&\checkmark&\checkmark\\
          Message&-&-&-&\checkmark\\
          Comment Count&-&\checkmark&\checkmark&-\\
          Comment List&-&\checkmark&\checkmark&-\\
          Parent&\checkmark&-&-&-\\
          OID&-&-&-&\checkmark\\
          Text&\checkmark&\checkmark&\checkmark&\checkmark\\
          \bottomrule
        \end{tabular}
      \end{table}

 In \textsf{S3} nodes are created for each of the requirements, issues, pull requests, and commits. 
The features  in Table~\ref{tab:artifactfeatures} are added as the node attributes. 
 Our prototype uses Neo4j\footnote{https://neo4j.com} as a graph database.

      \textsf{S4}  links the requirements to artifacts by creating edges between their associated nodes. 
      Two types of relationships are captured between the artifacts, namely \emph{tracesTo} and \emph{relatedCommit}. 
      The  \emph{tracesTo} relationship represents a trace link between a requirement node and a software development artifact node. 
      The \emph{relatedCommit} is a relation between a commit and a pull request node. 
      This relation captures provides insight into how the commits are organized by the team. 
      In practice, requirements can trace directly to commits or via pull requests (as seen in Figure˜\ref{fig:rawtracegraph}).

      We implement and evaluate three methods to extract trace links, which are represented with the \emph{tracesTo} relation. 
 The first method extracts keywords from the requirements and development artifacts and links the requirements to the artifacts that share keywords. 
 The other methods are based on the \textit{term frequency-inverse document frequency} (TF-IDF) vectors and \textit{word vectors} obtained from a pre-trained model.
For these methods, requirements are linked to the artifacts with similar vectors. 
      The overview of the processing of software artifacts to extract the trace links is shown in Algorithm~\ref{alg:process-software-artifacts}.
            \makeatletter
\algnewcommand\algorithmicswitch{\textbf{switch}}
\algnewcommand\algorithmiccase{\textbf{case}}
\algnewcommand\algorithmicassert{\texttt{assert}}
\algnewcommand\Assert[1]{\State \algorithmicassert(#1)}%
% New "environments"
\algdef{SE}[SWITCH]{Switch}{EndSwitch}[1]{\algorithmicswitch\ #1\ \algorithmicdo}{\algorithmicend\ \algorithmicswitch}%
\algdef{SE}[CASE]{Case}{EndCase}[1]{\algorithmiccase\ #1}{\algorithmicend\ \algorithmiccase}%
\algtext*{EndSwitch}%
\algtext*{EndCase}%
\makeatletter

\setphaserulewidth{0.4pt}

\begin{breakablealgorithm}
\caption{Trace links graph construction}
\label{alg:process-software-artifacts}
\begin{algorithmic}[1]
\State Input: $RSD$ \Comment{Requirement Specification Document}
\State Input: $GRU$ \Comment{Github Repository URL} 
\State Input: $M$ \Comment{Trace Extraction Method} 
\State Input: $\tau_{e}$ \Comment{Threshold for Vector-Based Methods} 
\State Output: $TG$ : \texttt{graph} \Comment{Trace Graph}
\phase{Fetch Software Artifacts}
% \LineComment{Request from Github graphQL API}
\State $\textit{issueList} \hspace{-0.1cm} \leftarrow$  \hspace{-0.2cm} getIssues($GRU$)\label{algl:m}
\State $\textit{prList} \hspace{-0.1cm} \leftarrow$  \hspace{-0.2cm} getPRs($GRU$)\label{algl:m}
\State $\textit{commitList} \hspace{-0.1cm} \leftarrow$  \hspace{-0.2cm} getCommits($GRU$)\label{algl:m}
% \LineComment{Parse directly from given Requirement Specification Document}
\State $\textit{reqList} \hspace{-0.1cm} \leftarrow$  \hspace{-0.2cm} getRequirements($RSD$)\label{algl:m}

\phase{Create Graph with Artifacts}

\State $\textit{sdaList} \hspace{-0.1cm} \leftarrow$  \hspace{-0.2cm} $issueList+prList+commitList$\label{algl:m}
% \State $\textit{TG} \hspace{-0.1cm} \leftarrow$  \hspace{-0.2cm} $issueNodes+prNodes+commitNodes+reqNodes$\label{algl:m}
\State $TG \leftarrow$  \texttt{graph} 

\For{\textbf{each} $a$ \textbf{in} sdaList} \label{algl:c}
\State $TG$.addNode(a)
\EndFor \label{algl:c}

\For{\textbf{each} $r$ \textbf{in} reqList} \label{algl:c}
\State $TG$.addNode(r)
\EndFor \label{algl:c}

\phase{Preprocess Artifacts for Trace Extraction}

\State lemmatize($reqList$, $method$)
\State lemmatize($sdaList$, $method$)
\If{$method=$ "vector-based"}
\State removeStopwords($reqList$, $method$)
\State removeStopwords($sdaList$, $method$)
\EndIf

\phase{Extract Trace Links}
\For{\textbf{each} $r$ \textbf{in} reqList}
\For{\textbf{each} $a$ \textbf{in} sdaList}
% \If{$method=$ "keyword"}
% \State $keywords \leftarrow$ extract(r)
% \If{$a.text$ contains any $kw$ in $keywords$}
% \State $TG$.addEdge(r,a)
% \EndIf
% \EndIf
% \If{$method=$ "vector-based"}
% \State r-v $\leftarrow$ createVector($r.text$)
% \State a-v $\leftarrow$ createVector($a.text$)
% \If{sim(r-v, a-v)  $\geq$ $\tau_{e}$}
% \State $TG$.addEdge(r,a)
% \EndIf
% \EndIf

\Switch{$method$}
\Case{keyword}
    \State $keywords \leftarrow$ extractKeywords(r)
    \If{contains($a.text, keywords$)}
        \State $TG$.addEdge(r,a)
    \EndIf
\EndCase
\Case{tf-idf}
    \State r-v $\leftarrow$ createTfidfVector($r.text$)
    \State a-v $\leftarrow$ createTfidfVector($a.text$)
    \If{sim(r-v, a-v)  $\geq$ $\tau_{e}$}
      \State $TG$.addEdge(r,a)
    \EndIf
\EndCase
\Case{word-vector}
    \State r-v $\leftarrow$ createWordVector($r.text$)
    \State a-v $\leftarrow$ createWordVector($a.text$)
    \If{sim(r-v, a-v)  $\geq$ $\tau_{e}$}
        \State $TG$.addEdge(r,a)
    \EndIf
\EndCase
\EndSwitch
\EndFor
\EndFor

\Return $TG$
\end{algorithmic}

\end{breakablealgorithm}

The \textit{getIssues, getPRs, getCommits} functions take a project repository  ($GRU$) and make API calls to the GitHub API\footnote{\url{https://docs.github.com/en/graphql}} to fetch a list of issues, PRs, and commits respectively. 
It fetches the properties shown in Table \ref{tab:artifactfeatures} for these software software artifacts. 
The \textit{$preprocess(list, method)$}  function takes  a list of software artifacts and a method for trace link creation. 
It lemmatizes the text property of each artifact in the list. 
If the method is vector-based then it also removes the stopwords. 
Thereafter, the trace links are determined according to desired method.
In the keyword based method shared keywords between the requirements and software artifacts are sought. In vector based methods the requirements and artifacts are vectorized and their similarities are compared. When similarities exceed a given threshold edges are formed.
For vector-based methods we utilize TF-IDF and word vectors.


% The $sEdge(req, sda, method)$  function determines whether a trace link exists between a given a requirement ($req$) and a software development artifact ($sda$) based on a trace link $method$. If  $method$ is `keyword extraction', the keywords are extracted from $req$ and $sda$ and an edge is created if they share keywords.
% If  $method$ is `vector-based' (tf-idf or word-embedding) vectors are created for  $req$ and $sda$  using their text property. 
% Then, the similarity between the vectors are calculated. 
% An edge is created if similarity value is above a predefined threshold.

 The following details the trace extraction methods referred to in the algorithm. 

      \paragraph{Keyword Case} To identify the most relevant keywords of the requirements, the following NLP methods are utilized:  
      % We identify the keywords from the requirements as our trace link direction is from requirements to the software development artifacts.
      \begin{itemize}
      \item \textit{Tokenization}: Each requirement is tokenized to obtain its words.
      \item  \textit{Part-of-Speech Tagging}: Each word is categorized according to its part-of-speech (POS) tags. 
      This work specifically focuses on nouns and verbs when identifying relevant keywords. 

      \item  \textit{Dependency parsing}: The dependency trees of the requirement sentences are obtained using spacy\footnote{https://spacy.io/api/dependencyparser} .
      Figure~\ref{fig:deptree} shows a dependency tree for a requirement.
      The verbs and nouns that are related to objects via the direct object and the object of preposition relations are used to create \emph{verb-object} and \emph{noun-object} pairs.
      The remaining verbs and nouns are also captured.
      All of these are used to find the artifacts that are relevant to the requirements.

      \item  \textit{Stopword removal}: English stopwords are used to remove the singleton nouns and verbs which are not distinguishing.

      \item  \textit{Project stopwords removal}: Users are allowed to provide project-specific stopwords. 
      \end{itemize}

\begin{figure*}[htbp]
    \centering
    \includegraphics[width=1\linewidth]{figs/displacy.png}
    \caption{The dependency tree of a requirement.}
    \label{fig:deptree}
  \end{figure*}

  Using these NLP tasks we identify the significant keywords from requirement specifications and prepare a base for identifying trace links. 
  Figure~\ref{fig:keywords} shows the extracted keywords for a given requirement.

  \begin{figure}[H]
    \centering
    \includegraphics[width=.96\linewidth]{figs/keywords.png}
    \caption{Keyword extraction from a requirement.}
    \label{fig:keywords}
  \end{figure}

  The extracted keywords are matched against the textual attributes of the software artifacts to determine the trace links.
  These traces are stored in a graph database with the relation type \emph{tracesTo}.

  \paragraph{TF-IDF Vectors} Initially, a set of all the words in the requirements and the software development artifacts is created. 
  Stopwords are removed from this set, from which
  % TF-IDF values are computed using the  
  TF-IDF vectors are created for each requirement and artifact using equations \ref{eq:tf}, \ref{eq:idf}, and \ref{eq:tfidf}. 
  The requirements are linked with artifacts that have a similarity score above a given threshold (see Section~\ref{sec:eval}).
  Figure~\ref{fig:tfidfvec} presents the steps for this method.

  \begin{align}
    TF(t,d) &= \frac{\text{frequency of t in d}}{\text{total number of terms in d}} \label{eq:tf} \\
    IDF(t) &= log\frac{N}{1+df} \label{eq:idf}\\
    TF\text{-}IDF(t,d) &= TF(t,d)*IDF(t) \label{eq:tfidf}
  \end{align}

      \begin{figure}[H]
        \centering
        \includegraphics[width=0.95\linewidth]{figs/tfidfvector2.png}
        \caption{Steps for extracting trace links based on TF-IDF vectors}
        \label{fig:tfidfvec}
      \end{figure}

      \paragraph{Word Vectors} To generate a vector for each artifact, we use a pre-trained word embeddings model (word2vec-google-news-300 \footnote{\url{https://huggingface.co/fse/word2vec-google-news-300}}). 
      We represent requirements as the average of their word vectors. 

      Just like in TF-IDF method, cosine similarity is used to link requirements to artifacts.
      The artifacts that have similarities above a predefined threshold are considered to be linked. 
      % To extract trace links for a requirement using TF-IDF and word vectors, 
      % we calculate the cosine similarity metric of the requirement's vector with the vectors of other artifacts. 
      Figure~\ref{fig:wordvec} presents the steps of the word vector method.

       \begin{figure}[htb]
        \centering
        \includegraphics[width=0.99\linewidth]{figs/wordvector.png}
        \caption{Steps of trace link extraction based on word vectors}
        \label{fig:wordvec}
      \end{figure}



\begin{figure}[htb]
    \centering
    \includegraphics[width=1\linewidth]{figs/rawTraceGraph.png}
    \caption{A segment of a trace graph.}
    \label{fig:rawtracegraph}
  \end{figure}

  The obtained trace links are added to the trace graph with the \emph{tracesTo} edge type. 
  The relations between pull requests and commit nodes are stored with the \emph{relatedCommit} edge type.
  Figure~\ref{fig:rawtracegraph} illustrates a segment of a trace graph of a requirement.

\begin{figure}[htb]
    \centering
    \includegraphics[width=.99\linewidth]{figs/traceGraph.png}
    \caption{A segment of a trace graph on a time axis.}
    \label{fig:tracegraph}
\end{figure}

The trace graph  not only visualizes the software development artifacts (SDA)  traced from requirements, but it also can display the lifetime of the activities over time. 
Figure~\ref{fig:tracegraph} shows the SDAs related to a requirement which are arranged along a time axis based on their \textit{creationDate} property. 
Here, we see that first pull request related to the requirement was created in week 8, while the issues concerning the planning of the requirement were created in the early stages of the development. 
Such visualization is useful during the development phase of software projects as well as a while performing retrospective analyses.




% - The requirement to be examined is the root.
% - The requirement is connected to its related SDAs with \textit{tracesTo} links.
% - Pull request nodes have related commit nodes connected with \textit{relatedCommit} link.

% Who looks at trace?\\
% Trace graph can help a project manager or a developer
% Why?\\
% What can be found?


% A project manager looking at the trace graph can identify:

% \begin{itemize}
%     \item The planning phase of this requirement goes back to week 1
%     \item The implementation of this feature started at week 8
%     \item This requirement took around 13 weeks of work??
%     \item ...
% \end{itemize}

% \pagebreak

% Or in the case of a problem or a bug related to annotation, for example, a developer can view the trace of the requirement about annotation, localizing the search for the problem. Lets say the problem is about updating annotations. Looking at the trace graph, an educated guess can be made, with the information about the problem, to look at a specific pull request. For example, in our case user can view \textit{"Create annotation model"} and \textit{"... add update and delete annotation endpoints..."} nodes. Both nodes can be further observed by using the url property, reaching the github page and examining the code related to them.



Finally, \textsf{S5} provides a visualization where the user can browse the software development artifacts based on the trace links. 
We report several types of information extracted from the repository to support software development project management. 
Neodash\footnote{\url{https://neo4j.com/labs/neodash/}} is integrated into our Neo4j graph database to provide an interactive dashboard to explore the trace graph.

The dashboard presents information about a software repository and enables exploration based on trace links.
First, an overview of the software artifacts for a software repository.  
The number of issues and pull requests is visualized in a stacked bar chart per week to view the project's progress over time. 
Similarly, the number of  issues closed per week is visualized with a stacked bar chart. 
The dashboard also shows the total number of open/closed issues, open/merged PRs, and the average number of trace links per requirement. 
These statistics serve as a snapshot of the current state of the project. 
Figure~\ref{fig:barcharts} shows the dashboard for a repository.

\begin{figure}[hbt]
    \centering
    \includegraphics[width=.9\linewidth]{figs/dashboard-barcharts.png}
    \caption{Information about the software artifacts of a project. The first bar chart shows the weekly issues and pull requests created. The second bar chart shows opened and closed artifacts per week.  On the right, is the total number of currently active issues and PRs and completed tasks (issues and PRs). }
    \label{fig:barcharts}
\end{figure}

The dashboard displays the trace links for a requirement over the course of the project on a weekly basis to track the progress of a requirement.
It is presented using a line chart as shown in Figure~\ref{fig:linechart}. 
The project manager can visualize the data of a single requirement or compare the progress of multiple requirements. 
This comparison allows the users to observe the effort associated with each requirement
since it displays the number of software development artifacts traced to them.

\begin{figure}[htb]
    \centering
    \includegraphics[width=.9\linewidth]{figs/linechart.png}
    \caption{The number of trace links for each requirement. }
    \label{fig:linechart}
\end{figure}

The dashboard shows a comparative  view of two requirements with their weighted relations to software artifacts using  a Sankey diagram (Figure~\ref{fig:sankey}).
Here, thicker lines represent a stronger relation between the requirements to their traced artifacts. 
Thus, one can see how requirements are related via their associated software artifacts.

\begin{figure}[htb]
    \centering
    \includegraphics[width=.9\linewidth]{figs/sankey.jpg}
    \caption{The weighted relations between requirements and their associated software artifacts.}
    \label{fig:sankey}
\end{figure}

Users can interact with the dashboard to gain insights on selected requirements as seen in Figure˜\ref{fig:perreq}. 
The identified traces are presented in graphical and tabular formats for  selected requirements.
Their  current status and weekly activities are displayed in the last section of the dashboard.

\begin{figure}[htb]
    \centering
    \includegraphics[width=.9\linewidth]{figs/perreq.jpg}
    \caption{Detailed information about a selected requirement.}
    \label{fig:perreq}
\end{figure}


\section{Evaluation}
\label{sec:eval}

This section presents the results of our preliminary evaluation and detailed evaluation plan for the future.

We evaluate our approach on a public GitHub repository of a group of computer engineering students for their software engineering course\footnote{Link Anonymized.}. The repository includes the project of an online learning platform. The requirements of the project are also in the repository written in a mixed format where some requirements are written as short phrases to describe a functionality whereas others as full shall statements. The requirements are structured hierarchically where a parent requirement is further refined into other requirement items.

The first two authors who are familiar with the project but not involved in the project manually extracted the trace links to serve as the ground truth. Below we report the performance of our keyword matching method in Table~\ref{tab:keyperf} and vector-based methods with various similarity thresholds in Table~\ref{tab:vecperf}.

\begin{table}[htb]
\centering
\caption{\label{tab:keyperf}Performance of the Keyword Matching Method}
\begin{tabular}{llll}
  \toprule
  Method & Recall & Precision & F1 Score \\ \midrule
  Keyword extraction & {0.865} & 0.212 & 0.340 \\
  \bottomrule
\end{tabular}
\end{table}

\begin{table}[htb]
\centering
\caption{Performance of the Vector-based Methods}
\label{tab:vecperf}
\begin{tabular}{lllllll}
   \toprule
    \multirow{2}{*}{\shortstack[l]{Similarity \\ Threshold}}
  & \multicolumn{3}{c}{Word-vector} &  \multicolumn{3}{c}{TF-IDF vector} \\
   \cmidrule{2-7}
                                                          & {Rec.} & {Prec.} & F1 &  {Rec.} & {Prec.} & F1\\
   \midrule
  0.05 & 1 & 0.043 & 0.082 & 0.839 & 0.121 & 0.211 \\
  0.15 & 1 & 0.043 & 0.082 & 0.573 & 0.256 & 0.354 \\
  0.25 & 1 & 0.043 & 0.082 & 0.244 & 0.43 & 0.311 \\
  0.35 & 1 & 0.043 & 0.082 & 0.095 & 0.392 & 0.153 \\
  0.45 & 0.965 & 0.071 & 0.132 & 0.025 & 0.125 & 0.042 \\
  0.55 & 0.865 & 0.1 & 0.179 & 0.013 & 0.121 & 0.023 \\
  0.65 & 0.294 & 0.3 & 0.297 & 0 & 0 & -- \\
   \bottomrule
 \end{tabular}
\end{table}

Based on the F1 scores, the TF-IDF vector-based method has the best F1 score followed by the keyword-matching method on this repository followed by keyword matching method. The performance of the vector-based methods  significantly varies according to the threshold value. Setting a low threshold links requirements with many artifacts yet few of these links are actually valid.

We do not reach any conclusions based on our preliminary evaluation. This evaluation demonstrates that our approach is applicable for tracing requirements in a software repository but we refrain to be conclusive on the performance of the methods. In the near future, we plan two additional evaluation studies.

The first study concerns evaluating our approach again in an educational setting where we analyze the repositories of student groups, report on the perceived usefulness and usability of our dashboard, and study any correlations between the statistics reported by our dashboard and the performance of the groups in the course.

The second study is a case study with an industry partner where we ask for the ground truth from the experts from the industry and report the performance of different methods to extract trace links as well as the perceived usefulness and usability of our dashboard in comparison to trace matrix, which is widely used in the industry for tracing requirements.

\section{Discussion}
\label{sec:discuss}

\emph{Observations.} Our experiments on the evaluation of the tool have led us to observe that the quality and consistency of the requirement statements and the software development artifacts (SDA) significantly impact the effectiveness of our approach in identifying trace links. Well-written requirement specifications that are self-explanatory and granulated lead to better performance of its traces along with software development artifacts that have textual data (e.g. title, description, comments) that is semantically related to the feature they implement.

\emph{Limitations.} The current implementation of our approach traces requirements to issues, pull requests, and commits. A more thorough trace should include other artifacts such as the architecture models, code, and test cases. Our implementation handles English only, other languages are not supported. The keyword matching method focuses on the syntax rather than the semantics of the word, yet consistent use of terms and agreeing on a glossary should mitigate this limitation.

\emph{Implications in practice.} Our approach aims to visualize software repositories based on trace links. Improving the performance of trace link extraction would provide more accurate data for the dashboard visualization. We hypothesize that software engineering educators and students can benefit from this visualization to track the progress of the software projects both during and after the development phase. Visualization of traces of a requirement presents the maturity of the implementation of a requirement and the contribution patterns of students. Similarly, companies can track the progress of their projects and can gain retrospective insights on projects for good and bad practices based on the visualizations provided in our dashboard.

\emph{Threats to validity.} Our evaluation presented in this paper is preliminary. To mitigate the threats to internal validity, we selected a repository the authors were not involved and two authors collaborated on building the ground truth to reduce personal bias. We cannot reach any conclusions about the external validity of the results. We need to conduct more studies to generalize the results. We share our replication package and invite the community to replicate our work for reliability.
\section{Conclusions}\label{section:conclusion}

% \begin{itemize}
%     \item Observations
%     \item Threads to validity
%     \item Future Work
% \end{itemize}

In this paper, we presented a tool that automates the traceability link recovery by identifying trace links between requirements written in natural language and software development artifacts(Issues, PRs, Commits) obtained from GitHub repositories. Through the implementation of three optional methods, namely keyword extraction, TF-IDF vector and word vector, we have demonstrated the capability of identifying traceability links of our tool. We have stored the traceability data we obtained in a graph database to visualize the trace links, and supported the data with an interactive dashboard, which provides users with comprehensive statistical insights.
Our experiments on the evaluation of the tool have led us to observe that the quality and consistency of the requirement statements and the software development artifacts(SDA) significantly impact the effectiveness of the tool's capability of identifying trace links. Well-written requirement specifications that are self-explanatory and granulated lead to easier identification of its traces along with SDA which have textual data (e.g. title, description, comments) that is semantically related to the feature they implement. In this manner, our tool could be utilized for educational purposes, to assist in the improvement of writing requirement specifications.

It is important to acknowledge the threats to the validity of our study. Firstly, the absence of a ground truth set that is established by analysts may introduce bias in our evaluation process. We have prepared the ground truth set and conducted our experiments on the student projects we have participated in. Thus, we do know the capabilities of our tool and the project that we experimented on. Future work should involve validation with industry professionals and experiments designed with a wider range of projects to enhance the operating scale of our tool. 

Other several avenues for future work include providing trace links between different kinds of software artifacts, such as from Issues to PRs, to enable a deeper view of the development process. Additionally, we plan to integrate different types of requirement specifications, to accommodate diverse requirement specification document formats. Furthermore, we intend to develop a product that fetches software artifacts from various platforms such as GitLab and Jira, to improve the compatibility of the tool with different development environments that are mostly preferred within the industry. Finally, we would like to expand the variety of methods used in identifying trace links based on the needs of the users and improve the accuracy.
% by increasing the recall and precision values obtained.

In conclusion, our tool provides an automated solution for requirement traceability(RT) and enhances the management of software projects. By optimizing the time and effort required for RT and offering educational opportunities for project management, our tool contributes to the overall success of software development projects. With future development and evaluation, our study carries the potential to positively impact traceability practices.



%We present a shared collaborative modeling environment instrumented with an automated glossary extraction pipeline from model labels using state-of-the-art NLP techniques without any supervision. Our approach prevents stakeholders from spending effort on a challenging and time-consuming task of manual glossary extraction. We develop two heuristics that use pre-trained BERT models and one simple heuristic that depends on substring matching.

 %In this section, we share our plans for glossary extraction. First, we plan to conduct an experimental evaluation with human experts as we stated in Sect. \ref{section:evaluation}. Second, we plan to improve our glossary extraction heuristics' performance using human experts' feedbacks. Finally, we aim to extend our method for automated domain model extraction from model labels and employing our method as a complementary for existing systems to form a hybrid system.










\bibliographystyle{IEEEtran}
\bibliography{paper}
\end{document}


%%% Local Variables:
%%% mode: latex
%%% TeX-master: t
%%% End:
